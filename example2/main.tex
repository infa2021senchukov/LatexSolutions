\documentclass[12pt]{article}

% report, book
%  Русский язык

%\usepackage{bookmark}
\usepackage{ bbold }

\usepackage[T2A]{fontenc}			% кодировка
\usepackage[utf8]{inputenc}			% кодировка исходного текста
\usepackage[english,russian]{babel}	% локализация и переносы
\usepackage[title,toc,page,header]{appendix}
\usepackage{amsfonts}
\usepackage{hyperref,bookmark}

\hypersetup{  % добавил Артем 
	pdftitle           = {Spectral theory of random matrices},
	pdfauthor          = {},
	pdfsubject         = {},
	pdfstartview       = {FitH},
	pdfborder          = {0 0 0},
	bookmarksopen      = true,
	bookmarksnumbered  = true,
	bookmarksopenlevel = 2,
	colorlinks = true,     linkcolor  = blue, % делает красивые цветные кликабеьные ссылки и сноски
	citecolor  = blue, filecolor  = blue,
	menucolor  = blue, urlcolor   = blue
}

% Математика
\usepackage{amsmath,amsfonts,amssymb,amsthm,mathtools,bm} 
%%% Дополнительная работа с математикой
%\usepackage{amsmath,amsfonts,amssymb,amsthm,mathtools} % AMS
%\usepackage{icomma} % "Умная" запятая: $0,2$ --- число, $0, 2$ --- перечисление

\usepackage{cancel}%зачёркивание
\usepackage{braket}
%% Шрифты
\usepackage{euscript}	 % Шрифт Евклид
\usepackage{mathrsfs} % Красивый матшрифт


\usepackage[left=2cm,right=2cm,top=1cm,bottom=2cm,bindingoffset=0cm]{geometry}
\usepackage{wasysym}

%размеры
\renewcommand{\appendixtocname}{Приложения}
\renewcommand{\appendixpagename}{Приложения}
\renewcommand{\appendixname}{Приложение}

\newcommand*\diff{\mathop{}\!\mathrm{d}} % добавил Артем
\newcommand*\Tr{\mathop{\mathrm{Tr}}} % добавил Артем
% Полная производная
\newcommand{\D}[2]{\frac{\diff{#1}}{\diff{#2}}}% добавил Артем
% Частная производная
\newcommand{\PD}[2]{\frac{\partial{#1}}{\partial{#2}}}% добавил Артем

\makeatletter
\let\oriAlph\Alph
\let\orialph\alph
\renewcommand{\@resets@pp}{\par
	\@ppsavesec
	\stepcounter{@pps}
	\setcounter{subsection}{0}%
	\if@chapter@pp
	\setcounter{chapter}{0}%
	\renewcommand\@chapapp{\appendixname}%
	\renewcommand\thechapter{\@Alph\c@chapter}%
	\else
	\setcounter{subsubsection}{0}%
	\renewcommand\thesubsection{\@Alph\c@subsection}%
	\fi
	\if@pphyper
	\if@chapter@pp
	\renewcommand{\theHchapter}{\theH@pps.\oriAlph{chapter}}%
	\else
	\renewcommand{\theHsubsection}{\theH@pps.\oriAlph{subsection}}%
	\fi
	\def\Hy@chapapp{appendix}%
	\fi
	\restoreapp
}


\makeatother
\newtheorem{theorem}{Теорема}[section]
\newtheorem{predl}[theorem]{Предложение}
\newtheorem{sled}[theorem]{Следствие}

\theoremstyle{definition}
\newtheorem{zad}{Problem}[section]
\newtheorem{upr}[zad]{Упражнение}
\newtheorem{defin}[theorem]{Определение}


\begin{document}


    
 \section{Конформная симметрия многомерных пространств}
 \begin{defin}
Конформное преобразование --- обратимое преобразование координат, сохраняющее метрический тензор с точностью до растяжений.
\end{defin}
\begin{equation}
g_{\mu\nu}'(x')= \frac{\partial x^\alpha}{\partial x'^\mu}\frac{\partial x^\beta}{\partial x'^\nu}g_{\alpha\beta}(x)=\Lambda(x)g_{\mu\nu}(x)
\end{equation}
Рассмотрим преобразования метрического тензора в случае инфинитизимальных преобразований:
\begin{equation}
x'^\mu=x^\mu+\varepsilon^\mu(x)
\end{equation}
Подставляя в формулу $(1)$ получаем (при $d=1$ $\varepsilon$ - любая, при $d=2$ - Коши-Риман):
\begin{equation}
\partial_\mu\varepsilon^\nu + \partial_\nu\varepsilon^\mu = f(x)g_{\mu\nu}
\end{equation}
Где $f(x)$ считаем, взяв след обоих частей $(3)$:
\begin{equation}
f(x)=\frac{2}{d}\partial_\mu\varepsilon^\mu
\end{equation}
Далее везде полагаем $g_{\mu\nu}= \eta_{\mu\nu}$. Дифференцируя $(3)$, получаем:
\begin{equation}
2\partial_\mu\partial_\nu\varepsilon_\rho + \partial_\nu\varepsilon^\mu = \eta_{\mu\rho}\partial_\nu f
\end{equation}
Сворачивая с $\eta^{\mu\nu}$ получаем:
\begin{equation}
2\partial^2\varepsilon_\mu=(2-d)\partial_\mu f
\end{equation}
Дифференцируя и используя $(3)$:
\begin{equation}
(2-d)\partial_\mu\partial_\nu f = \eta^{\mu\nu}\partial^2 f
\end{equation}
Сворачивая с $\eta^{\mu\nu}$ :
\begin{equation}
(d-1)\partial^2 f =0
\end{equation}
Рассмотри случай $d>2$. Из $(7)$ получаем: $\partial_\mu\partial_\nu f = 0$. Значит, используя $(6)$:
\begin{equation}
f = A + B_\mu x^\mu \\
\end{equation}
\begin{equation}
\varepsilon^\mu = a_mu + b_{\mu\nu}x^\nu+c_{\mu\nu\rho}x^\nu x^\rho \quad c_{\mu\nu\rho} = c_{\mu\rho\nu}
\end{equation}
$(3)$ и $(4)$ дают уравнения на $b_{\mu\nu}$:
\begin{equation}
b_{\mu\nu} = \alpha\eta_{\mu\nu}+m_{\mu\nu} \quad m_{\mu\nu} = -m_{\nu\mu}
\end{equation}
$(5)$ на $c_{\mu\rho\nu}$:
\begin{equation}
c_{\mu\nu\rho} = \eta_{\rho\nu}b_\nu + \eta_{\mu\nu}b_\rho - \eta_{\rho\nu}b_\mu \quad b_{\mu} = \frac{1}{d}c^\nu_{\nu\mu}
\end{equation}
Тогда $x_\mu$ инфинитизимально преобразуется как:
\begin{equation}
x'^\mu = x^\mu +2(x\cdot b)x^\mu - b^\mu x^2
\end{equation}
Интегрируя, а также пользуясь выражением для $\varepsilon^\mu$ получаем:
  	\begin{center}
			\begin{tabular}{|c|c|c|}
				\hline
       & эл-т группы & генератор        \\ \hline
				трансляции & $x'^\mu = x^\mu + a^\mu$  &   $P_\mu=-i\partial_\mu$    \\ \hline
    растяжения & $x'^\mu = \alpha x^\mu$  & $D=-ix^\mu\partial_\mu$      \\ \hline
     повороты & $x'^\mu = M^\mu_\nu x^\nu$  & $L_{\mu\nu}=i(x^\mu\partial_\nu-x^\nu\partial_\mu)$      \\ \hline
     спец. конф. преобр. & $x'^\mu = \frac{x^\mu-b^\mu x^2}{1-2(b\cdot x)+b^2x^2}$  & $K_{\mu}=-i(x^\mu x^\nu\partial_\nu-x^2\partial_\mu)$      \\ \hline
			\end{tabular}
\end{center}
Переобозначая:
\begin{equation}
J_{\mu\nu} = L_{\mu\nu} \quad J_{-1,0} = D \quad J_{-1,\mu} = \frac{1}{2}(P_\mu-K_\mu)\quad J_{0,\mu} = \frac{1}{2}(P_\mu+K_\mu)
\end{equation}
Получим коммутационные соотношения, доказывающие изоморфизм конформной группы размерности $d$ и $SO(d+1,1)$:
\begin{equation}
[J_{ab},J_{cd}] = i(\eta{ab}J_{bc}+\eta{bc}J_{ad}-\eta{ac}J_{bd}-\eta{bd}J_{ac})
\end{equation}
Замети
	\section{ Конформная симметрия в двух измерениях}
В двух измерениях уравнение $(3)$ принимает вид:
\begin{equation}
\partial_0\varepsilon_0 = \partial_1\varepsilon_1 \quad \partial_0\varepsilon_1 = -\partial_1\varepsilon_0
\end{equation}
Что соответствует условиям Коши - Римана для функции $\varepsilon(x)$. Перейдя в комплексные координаты , можно заметить, что все инфинитизимальные преобразования задаются некоторой голоморфной функцией $f(z)$:
\begin{equation}
z'=z+f(z)\quad
\end{equation}
  \section{Алгебра Витта и ее центральное расширение алгебра Вирасоро.}
 Пусть в общем случае $f(z)$ - мероморфная функция, имеющая полюса за пределами некоторого открытого множества, тогда в окресности нуля она раскладывается в ряд Лорана.
 \begin{equation}
z'=z+\varepsilon(z)=z+\sum_{n\in\mathbb{Z}}\varepsilon_n(-z^{n+1}) 
\end{equation}
Данное инфинитизимальное преобразование определяет генераторы некоторой алгебры Ли:
 \begin{equation}
l_n = -z^{n+1}\partial_z
\end{equation}
Их количество бесконечно, а коммутационные соотношения следующие:
\begin{equation}
[l_m,l_n] = (m-n)l_{m+n}
\end{equation}
 \begin{defin}
Алгебру Ли с бесконечным числом элементов, удовлетворяющих данным комутационным соотношениям называют алгеброй Витта.
\end{defin}
Рассмотрим центральное расширение построенной алгебры. Дополним ее некоторой комплексной константой $c$, коммутирующей со всеми элементами, но влияющую на коммутационные соотношения.
\begin{equation}
[L_m,L_n] = (m-n)L_{m+n}+cp(n,m)
\end{equation}
Так как коммутатор обязан быть антисимметричным и удовлетворять тождеству Якоби, получим ограничение на функцию $p(n,m)$:
\begin{equation}
p(n,m)=\frac{1}{12}(n+1)n(n-1)\delta_{m,-n}
\end{equation}
Использована нормировка $p(2,-2)=\frac{1}{2}$.
 \begin{defin}
Алгебра Вирасоро - центральное расширение алгебры Витта с коммутационными соотношениями:
\begin{equation}
[L_m,L_n] = (m-n)L_{m+n}+\frac{c}{12}(m^3-m)\delta_{m,-n}
\end{equation}
\end{defin}
  \section{Структура группы Мёбиуса $SL(2,R)$.}
  Очевидно, что группа Мебиуса всех дробно линейных преобразований комплексной плоскости изоморфна группе $SL(2,R)$/\{$\pm$ 1\}. По теореме о жордановой нормальной форме любую матрицу $2\times2$ можно представить в одном из следующих видов:
 \begin{equation}
 1.
\begin{pmatrix}
\lambda & 1\\
0 & \lambda
\end{pmatrix}
\quad z'= z + \frac{1}{\lambda}
\end{equation}
- параболическое преобразование
 \begin{equation}
 1.
\begin{pmatrix}
\lambda & 0\\
0 & \mu
\end{pmatrix}
\quad z'= cz
\end{equation}
- эллиптическое в случае $|c|=1$
- гиперболическое в случае $ c \in\mathbb{R}$ и $c>0$. 
Других преобразований в группе Мебиуса нет.
\end{document}