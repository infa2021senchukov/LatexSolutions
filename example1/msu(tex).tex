\documentclass[12pt]{article}

% report, book
%  Русский язык

%\usepackage{bookmark}
\usepackage{ bbold }

\usepackage[T2A]{fontenc}			% кодировка
\usepackage[utf8]{inputenc}			% кодировка исходного текста
\usepackage[english,russian]{babel}	% локализация и переносы
\usepackage[title,toc,page,header]{appendix}
\usepackage{amsfonts}
\usepackage{hyperref,bookmark}

\hypersetup{  % добавил Артем 
	pdftitle           = {Spectral theory of random matrices},
	pdfauthor          = {},
	pdfsubject         = {},
	pdfstartview       = {FitH},
	pdfborder          = {0 0 0},
	bookmarksopen      = true,
	bookmarksnumbered  = true,
	bookmarksopenlevel = 2,
	colorlinks = true,     linkcolor  = blue, % делает красивые цветные кликабеьные ссылки и сноски
	citecolor  = blue, filecolor  = blue,
	menucolor  = blue, urlcolor   = blue
}

% Математика
\usepackage{amsmath,amsfonts,amssymb,amsthm,mathtools,bm} 
%%% Дополнительная работа с математикой
%\usepackage{amsmath,amsfonts,amssymb,amsthm,mathtools} % AMS
%\usepackage{icomma} % "Умная" запятая: $0,2$ --- число, $0, 2$ --- перечисление

\usepackage{cancel}%зачёркивание
\usepackage{braket}
%% Шрифты
\usepackage{euscript}	 % Шрифт Евклид
\usepackage{mathrsfs} % Красивый матшрифт


\usepackage[left=2cm,right=2cm,top=1cm,bottom=2cm,bindingoffset=0cm]{geometry}
\usepackage{wasysym}

%размеры
\renewcommand{\appendixtocname}{Приложения}
\renewcommand{\appendixpagename}{Приложения}
\renewcommand{\appendixname}{Приложение}

\newcommand*\diff{\mathop{}\!\mathrm{d}} % добавил Артем
\newcommand*\Tr{\mathop{\mathrm{Tr}}} % добавил Артем
% Полная производная
\newcommand{\D}[2]{\frac{\diff{#1}}{\diff{#2}}}% добавил Артем
% Частная производная
\newcommand{\PD}[2]{\frac{\partial{#1}}{\partial{#2}}}% добавил Артем

\makeatletter
\let\oriAlph\Alph
\let\orialph\alph
\renewcommand{\@resets@pp}{\par
	\@ppsavesec
	\stepcounter{@pps}
	\setcounter{subsection}{0}%
	\if@chapter@pp
	\setcounter{chapter}{0}%
	\renewcommand\@chapapp{\appendixname}%
	\renewcommand\thechapter{\@Alph\c@chapter}%
	\else
	\setcounter{subsubsection}{0}%
	\renewcommand\thesubsection{\@Alph\c@subsection}%
	\fi
	\if@pphyper
	\if@chapter@pp
	\renewcommand{\theHchapter}{\theH@pps.\oriAlph{chapter}}%
	\else
	\renewcommand{\theHsubsection}{\theH@pps.\oriAlph{subsection}}%
	\fi
	\def\Hy@chapapp{appendix}%
	\fi
	\restoreapp
}


\makeatother
\newtheorem{theorem}{Теорема}[section]
\newtheorem{predl}[theorem]{Предложение}
\newtheorem{sled}[theorem]{Следствие}

\theoremstyle{definition}
\newtheorem{zad}{Problem}[section]
\newtheorem{upr}[zad]{Упражнение}
\newtheorem{defin}[theorem]{Определение}


\begin{document}


    
 \section{Элементы теории представлений алгебры $\mathfrak{sl_2}$}
 \begin{defin}
$\mathfrak{sl_2}(\mathbb{C})$ --- это алгебра Ли матриц 2 × 2 с нулевым следом.
\end{defin}

Алгебру Ли $\mathfrak{sl_2}(\mathbb{C})$ можно задать стандартными образующими $e$, $f$, $h$ со следующими коммутационными соотношениями:
\begin{equation}
    [e, f] = h\quad
 [h, e] = 2e\quad
 [h, f] = -2f
\end{equation}

Пусть $(V, \rho)$ --- конечномерное представление алгебры Ли $\mathfrak{sl_2}(\mathbb{C})$, то есть $V$ --- конечномерное векторное пространство, а $\rho(g)$ --- линейные операторы, действующие на нём, $g \in \mathfrak{sl_2}(\mathbb{C})$.


    Пусть $v \in V$ --- собственный вектор оператора $\rho(h)$ с собственным значением $\mu$. Из коммутационных соотношений получается, что вектор $\rho(e) v$ --- собственный для $\rho(h)$ с собственным значением $\mu + 2$, а вектор $\rho(f) v$ --- собственный для $\rho(h)$ с собственным значением $\mu - 2$.


    Так как ввиду конечномерности данная цепочка должна обрываться, существует собственный для $\rho(h)$ вектор $v \in V$ такой, что $\rho(e) v = 0$. Такие вектора называются особыми. Пусть $v_{\lambda}$ --- особый вектор. Можно построить цепочку собственных векторов для $\rho(h)$, последовательно действуя на $v_{\lambda}$ оператором $\rho(f)$, а точнее рассмотреть подпространство $V_{\lambda}$, натянутое на вектора вида $\rho(f) ^ k v_{\lambda}$. $V_{\lambda}$ является подпредставлением в $V$ и конечномерно тогда и только тогда, когда $\lambda \in \mathbb{Z}_{\ge 0}$.



Таким образом, всякое неприводимое представление изоморфно $V_n$ для некоторого $n \in \mathbb{Z}_{\ge 0}$. Так как $dim V_n = n + 1$, то неприводимое представление данной размерности единственно с точностью до изоморфизма.


	\section{Представления алгебры Вирасоро, модули Верма и их характеры}
    Алгебра Вирасоро задается коммутационными соотношениями:
    	 \begin{equation}
 	    [L_n,L_m]=(n-m)L_{n+m}+\frac{c}{12}n(n^2-1)\delta_{n+m,0}.
 	\end{equation}	
  Построим ее представление по аналогии с тем, как было построено представление алгебры $su(2)$ в квантовой механике (так же будем использовать бра и кет обозначения). \\
  Так как ни одна пара генераторов не коммутирует, перейдем в базис одного из них - $L_0$.  Обозначим за $\ket{h}$ состояние со старшим весом с собственным значением $h$. Так как:
  	\begin{equation}
 	    [L_0,L_m]=-mL_{m} \quad m >0 
 	\end{equation}	
Точно так же как и в алгебре $su(2)$, $L_m$ - понижающий оператор, а  $L_{-m}$ - повышающий. Мы должны принять:
 	\begin{equation}
 	    L_n\ket{h}=0 \quad n >0, 
 	\end{equation}
  что соответствует физическому смыслу $L_{n}$ как операторов, действующих на примарные поля, если положить: 
  \begin{equation}
 	    \ket{h} = \Phi_{h}(0) \ket{0}\quad
            \bra{h} =\bra{0} \Phi(\infty), 
 	\end{equation}
        где  $\ket{0}$ - состояние вакуума.\\
        Рассмотрим оператор $ L_{\bold{-k}} = L_{-k_1}...L_{-k_n}$:
         Так как $L_{-m}$ - повышающий оператор, $ L_{\bold{-k}}\ket{h}$ - собственный вектор оператора $L_0$. Все остальные базисные вектора данного пространства получаются действием $L_{\bold{-k}}\ket{h}$ (применением повышающего оператора всеми возможными способами)\\
         \textbf{Определение} Модуль Верма $V_{h,c}$ - представление алгебры Вирасоро, заданное следующими соотношениями $\{  L_{\bold{-k}}\ket{h}: 1\geq k_1\geq k_2\geq...,  L_n\ket{h}=0 \quad n >0,  L_0\ket{h}=h\ket{h} \}$
         Построенное таким образом представление распадается на прямую сумму конечномерных подпространств $V_{h, N}$:
           \begin{equation}
 	    V_{h, N} = span\{L_{\bold{-k}}\ket{h}:k_1+...+k_n=N\}
 	\end{equation}
  Данные подпространства являются собственными для $L_0$:
         \begin{equation}
 	   L_0L_{\bold{-k}}\ket{h}=(h+N)L_{\bold{-k}}\ket{h}
 	\end{equation}
  В качестве примера можно привести таблицу нескольких первых состояний $N$:
  	\begin{center}
			\begin{tabular}{|c|c|}
				\hline
      0 & $\ket{h}$         \\ \hline
				1 & $L_{-1}\ket{h}$         \\ \hline
    2 & $L_{-2}\ket{h}$, $L_{-1}^2\ket{h}$        \\ \hline
     3 & $L_{-3}\ket{h}$, $L_{-1}L_{-2}\ket{h}$, $L_{-1}^3\ket{h}$         \\ \hline
			\end{tabular}
\end{center}
Введем функцию характера модуля Верма:
           \begin{equation}
 	   \chi(t) = Tr(t^{L_0-\frac{c}{24}})|_{V_h}=\sum_{N=0}^{\infty}Tr(t^{L_0-\frac{c}{24}})|_{V_{h,N}}=\sum_{n=0}^{\infty}dim(n+h)q^{n+h-\frac{c}{24}}
 	\end{equation}
  Таким образом, фуеция характера - производящая функция размерности каждого подпространства. Так как $dim V_{h,N}=p(N)$, то, пользуясь следующими тождествами для функции Эйлера:
             \begin{equation}
 	  \frac{1}{\varphi(t)}=\prod_{n=1}^{\infty}\frac{1}{1-q^n}=\sum_{n=0}^{\infty}p(n)t^n,
 	\end{equation}
  получаем:
   \begin{equation}
 	  \chi(t) = \frac{t^{h-\frac{c}{24}}}{\prod_{n=1}^{\infty}(1-t^k)}
 	\end{equation}
  \section{Вырожденные представления}
  В зависимости от значений $h$, $c$ в построенном модуле Верма могут находиться вектора с нулевой или отрицательной нормой, в случае унитарной теории они должны быть удалены из представления. Для их нахождения приравняем к нулю определитель Каца - определитель матрицы со следующими матричными элементами: 
     \begin{equation}
 	  \bra{h}\prod_{i}L_{k_i}\prod_{j}L_{-m_j}\ket{h}\quad \sum_{i}k_i=\sum_{j}k_j=N
     \end{equation}
     Существует общая формула:
          \begin{equation}
 	  det M_N=\alpha_N\prod_{p,q\leq N, p,q> 0}(h-h_{p,q}(c))^{P(N-pq)}\end{equation} 
    \begin{equation*}
    h_{p,q}(m)=\frac{((m+1)p-mq)^2-1}{4m(m+1)} \quad m = -\frac{1}{2}\pm\frac{1}{2}\sqrt{\frac{25-c}{1-c}}
     \end{equation*}
     Если при каком-то значении $n$ вектор $\ket{h+n}$ оказался равным нулю, то существует $P(N-n)$ нулевых состояний.
     
\end{document}